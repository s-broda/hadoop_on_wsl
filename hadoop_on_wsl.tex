\documentclass[11pt,a4paper]{article}
\usepackage[T1]{fontenc}
\usepackage{exscale,latexsym,amsmath,amsfonts,amsthm,graphicx,natbib,times}
\usepackage[mtpcal,mtphrb,subscriptcorrection]{mtpro2}%either this or the next line must be commented out
\usepackage{fullpage}
\usepackage[usenames]{color}
\usepackage[pagebackref,colorlinks=true,breaklinks,linkcolor=blue,citecolor=blue,urlcolor=blue,bookmarks,pdfpagemode=UseOutlines,a4paper,pdftitle={Installing Hadoop on Windows Subsystem for Linux}]{hyperref}
\title{Installing Hadoop on the Windows Subsystem for Linux}
\author{Simon A.\ Broda\\\href{mailto:simon.broda@uzh.ch}{simon.broda@uzh.ch}}

\begin{document}
\maketitle
To run \href{https://hadoop.apache.org/}{Hadoop} under Microsoft Windows, you can either \href{https://wiki.apache.org/hadoop/Hadoop2OnWindows}{build from source} or \href{https://github.com/MuhammadBilalYar/Hadoop-On-Window/wiki/Step-by-step-Hadoop-2.8.0-installation-on-Window-10}{follow this guide} to use the windows binaries (not tested). On Windows 10, a third option is to use the Windows Subsystem for Linux (WSL).
To do this, install WSL following the instructions  \href{https://docs.microsoft.com/en-us/windows/wsl/install-win10}{here}. Below, it is assumed that you chose Ubuntu; the commands will be different for other distributions.
\begin{itemize}
\item The first step is to install Java in Ubuntu; to do that, open bash from the start menu (or from search). Bash is the \emph{shell} for Ubuntu on WSL. In it, you do

\verb+sudo apt-get install default-jdk+

\item After that, you can follow \href{https://jonboulineau.me/blog/hadoop/hadoop-on-wsl}{this guide}: start by doing
\begin{verbatim}
sudo apt purge openssh-server
sudo apt install openssh-server
sudo service ssh start
\end{verbatim}
to fix an issue with SSH.

\item Then, do

\verb+update-alternatives --config java+

and write down the path that it gives you, up to but not including the \verb+jre/bin/java+ at the end. For me, this gives \verb+/usr/lib/jvm/java-8-openjdk-amd64+.

\item We will set up Hadoop for pseudo-distributed operation, following \href{https://hadoop.apache.org/docs/stable/hadoop-project-dist/hadoop-common/SingleCluster.html}{this guide}: start by installing some required packages by doing
\begin{verbatim}
sudo apt-get install ssh
sudo apt-get install rsync
\end{verbatim}

\item Then, download Hadoop:
\begin{verbatim}
wget http://www-eu.apache.org/dist/hadoop/common/\
hadoop-2.9.1/hadoop-2.9.1.tar.gz
\end{verbatim}
(the backslash is a line continuation character in bash; this is so the command fits on this page. You can also delete the backslash and put the whole command on one line).
\item Unpack it:
\begin{verbatim}
tar -xvzf hadoop-2.9.1.tar.gz
\end{verbatim}
\item Next, you do
\begin{verbatim}
cd hadoop-2.9.1
nano etc/hadoop/hadoop-env.sh
\end{verbatim}
This opens an editor (nano). In nano, using  \verb+ctrl-o+ to save, \verb+ctrl-x+ to exit, change the line
\begin{verbatim}
export JAVA_HOME=${JAVA_HOME}
\end{verbatim}
so that the path equals the one you wrote down earlier. Be careful to put a slash and no space before the path; for me, this yields
\begin{verbatim}
export JAVA_HOME=/usr/lib/jvm/java-8-openjdk-amd64
\end{verbatim}

\item It remains to configure Hadoop for pseudo-distributed operation. Again using nano,
\begin{verbatim}
nano etc/hadoop/core-site.xml
\end{verbatim}
In that file, change

\begin{verbatim}
<configuration>

</configuration>
\end{verbatim}
to
\begin{verbatim}
<configuration>
    <property>
        <name>fs.defaultFS</name>
        <value>hdfs://localhost:9000</value>
    </property>
</configuration>
\end{verbatim}
(you should be able to paste into the bash window by just right-clicking).
\item Finally, do
\begin{verbatim}
nano etc/hadoop/hdfs-site.xml
\end{verbatim}
and modify
\begin{verbatim}
<configuration>

</configuration>
\end{verbatim}
to
\begin{verbatim}
<configuration>
    <property>
        <name>dfs.replication</name>
        <value>1</value>
    </property>
</configuration>
\end{verbatim}

\item That should pretty much be it. You should now be able to do
\begin{verbatim}
bin/hdfs namenode -format
sbin/start-dfs.sh
\end{verbatim}
and then point your browser at \verb+http://localhost:50070+. Note that while Hadoop starts up, you may be prompted several times to confirm whether you are sure that you want to connect; just enter \verb+yes<enter>+. The default is no, so don't just hit \verb+<enter>+. You may also need to enter your password several times. You can ignore the warning about not being able to set the \texttt{niceness}.
\item To shut down Hadoop, do
\begin{verbatim}
sbin/stop-dfs.sh
\end{verbatim}
\item Note that the next time you open bash, you have to do
\begin{verbatim}
cd hadoop-2.9.1
\end{verbatim}
for the above commands to work again without modification.
\end{itemize}
\end{document}
